\chapter{Introduction}\label{ch:introduction}
The relationship between physics and finance is a long-standing one. \cite{bachelier} is perhaps the first example of a model developed by physicists, in this case Brownian motion, being applied to financial markets, specifically to the pricing of derivative products. This work would remain largely unnoticed until \cite{black_scholes} was published, which is now considered the foundation of modern quantitative finance.

Beyond derivative pricing, econophysics as a field has been active since the 1990s, with the aim of applying methods from statistical physics to a wide range of problems in economics and finance. \cite{bouchaud_mezard_2000}, for instance, examines the distribution of wealth in a simplified model of an economy, mapping this problem to the random `directed polymer' problem.

For an introduction to the field of econophysics, refer to  \cite{Mantegna_Stanley_1999} and \cite{econophysics_2011_review}.

\section{Motivation}
The scope of this work is to review the spin model of financial price introduced in \cite{bornholdt}, after going through the relevant core background in statistical physics. From simulations (\textcolor{red}{and analytical results?}), we will see that the price time-series generated by this model exhibits properties similar to those observed in real financial data, challenging the assumptions of commonly used financial models. 