\chapter{Bornholdt model}\label{ch:chapter2}
We now present the model proposed by Bornholdt in \cite{bornholdt}. The idea is to formulate a model with maximum simplicity, which includes the possibility of strategic interaction in the market. The model is based on the Ising model, which is a model of ferromagnetism in statistical mechanics.

\section{The Ising model}
The Ising model is a simple mathematical model of ferromagnetic materials. In its description, we will mostly follow the notation and tools presented in \cite{mezard_book}. It consists of Ising spins (that is, spins which can take binary values) on a d-dimensional cubic lattice. Mathematically, given a cubic lattice $\mathbb{L}=\{1,\dots,L\}^n$, we define an an Ising spin $\sigma_i\in\{-1,1\}$ for each site $i\in\mathbb{L}$. Then, we can have any configuration $\underline{\sigma} = (\sigma_1,\dots,\sigma_n) \in \mathcal{X}_N=\{+1,-1\}^{\mathbb{L}}$. The energy of a configuration $\underline{\sigma}$ is given by:
\begin{equation}
    H(\underline{\sigma}) = -\sum_{\langle i,j\rangle}\sigma_i\sigma_j - B\sum_i \sigma_i
\end{equation}
Where the sum over $\langle i,j\rangle$ is a sum over all nearest neighbors, and $B$ is an external magnetic field. The probability of a configuration $\underline{\sigma}$ is given by the Boltzmann distribution:
\begin{equation}
    P(\underline{\sigma}) = \frac{e^{-\beta H(\underline{\sigma})}}{Z}
\end{equation}
Where $\beta$ is the inverse temperature, and $Z$ is the partition function:
\begin{equation}
    Z = \sum_{\underline{\sigma}\in\mathcal{X}_N}e^{-\beta H(\underline{\sigma})}
\end{equation}