\chapter{Bornholdt model}\label{ch:chapter2}
\textcolor{red}{Add motivation and intuition}\\
We now present the model proposed by Bornholdt in \cite{bornholdt}. The idea is to formulate a model with maximum simplicity, which includes the possibility of strategic interaction in the market. The model is based on the Ising model, which is a model of ferromagnetism in statistical mechanics.

\section{Theoretical background: the Ising model}
\textcolor{red}{Needs to be clearer}\\
The Ising model is a simple mathematical model of ferromagnetic materials. In its description, we will mostly follow the notation and tools presented in \cite{mezard_book} and from professor Mézard's lecture notes. It consists of Ising spins (that is, spins which can take binary values) on a d-dimensional cubic lattice. Mathematically, given a cubic lattice $\mathbb{L}=\{1,\dots,L\}^n$, we define an an Ising spin $\sigma_i\in\{-1,1\}$ for each site $i\in\mathbb{L}$. Then, we can have any configuration $\underline{\sigma} = (\sigma_1,\dots,\sigma_n) \in \mathcal{X}_N=\{+1,-1\}^{\mathbb{L}}$. The energy of a configuration $\underline{\sigma}$ is given by:
\begin{equation}
    H(\underline{\sigma}) = -\sum_{\langle i,j\rangle}\sigma_i\sigma_j - B\sum_i \sigma_i
\end{equation}
Where the sum over $\langle i,j\rangle$ is a sum over all nearest neighbors, and $B$ is an external magnetic field. At equilibrium, the probability of a configuration $\underline{\sigma}$ is given by the Boltzmann distribution:
\begin{equation}
    P(\underline{\sigma}) = \frac{e^{-\beta H(\underline{\sigma})}}{Z}
\end{equation}
Where $\beta$ is the inverse temperature, and $Z$ is the partition function:
\begin{equation}
    Z = \sum_{\underline{\sigma}\in\mathcal{X}_N}e^{-\beta H(\underline{\sigma})}
\end{equation}
Interestingly, despite its simplicity, an analytical solution has been found only in the $d=1$ and $d=2$ cases. Higher dimensions remain unsolved, but numerical methods can be used to study the model in these cases.

One important quantity in the Ising model is the magnetization, which is defined as:
\begin{equation}
    m = \frac{1}{N}\sum_i \langle \sigma_i \rangle
\end{equation}
where $\langle \cdot \rangle$ denotes the average.

\subsection{Solution of the Ising model in the one-dimensional case}
For simplicity, assume $B=0$. In the one-dimensional case, the Ising model can be solved exactly. Recall that:
\begin{equation}
    H(\underline{\sigma}) = -\sum_{\langle i,j\rangle}\sigma_i\sigma_j
\end{equation}
Then, the partition function is given by:
\begin{equation}
    Z = \sum_{\underline{\sigma}\in\mathcal{X}_N}e^{-\beta H(\underline{\sigma})} =
    \sum_{\underline{\sigma}\in\mathcal{X}_N}e^{\beta\sum_{\langle i,j\rangle}\sigma_i\sigma_j}
\end{equation}
Since each spin is connected to its nearest neighbors, we can write:
\begin{equation}
    Z =  \sum_{\underline{\sigma}\in\mathcal{X}_N}e^{\beta\sum_n\sigma_n\sigma_{n+1}}
\end{equation}
Let us define $\tau_n = \sigma_{n-1}\sigma_{n} \implies \sigma_n = \tau_n\tau_{n-1}\dots\tau_2\sigma_1$. Then, we can write:
\begin{equation}
    \begin{gathered}
    Z =  \sum_{\sigma_1\in\{-1,1\}}\sum_{\tau_2,\dots\tau_n}e^{\beta\sum_n\tau_n}
    = 2\sum_{\tau_2,\dots,\tau_N}e^{\beta\sum_n\tau_n}\\
    = 2\sum_{\tau_2,\dots,\tau_N}\prod_n e^{\beta\tau_n}
    = 2(\sum_{\tau_2}e^{\beta\tau_2})\dots(\sum_{\tau_N}e^{\beta\tau_N})\\
    = 2(2\cosh(\beta))^{N}
    \end{gathered}
\end{equation}
Thus, we have found an analytical expression for the partition function in the one-dimensional case. The magnetization can be computed as:
\begin{equation}
    m = \frac{1}{N}\sum_i \langle \sigma_i \rangle = \frac{1}{N}\frac{\partial}{\partial \beta}\log Z = \tanh(\beta)
\end{equation}
%check magnetization formula

% \subsection{The Curie-Weiss model}
% The Curie-Weiss model is remarkably similar to the Ising model, but with all the spins interacting with each other, then, the model represents a fully connected graph of spins rather than a lattice. The Hamiltonian is given by:
% \begin{equation}
%     H(\underline{\sigma}) = -\frac{1}{N}\sum_{i,j}\sigma_i\sigma_j - B\sum_i \sigma_i
% \end{equation}
% The scaling factor $\frac{1}{N}$ is introduced to have a non-trivial free energy. This model is interesting because it introduced the concept of mean-field approximations. To compute the partition function, we first notice that the empirical magnetization is:
% \begin{equation}
%     m(\underline{\sigma}) = \frac{1}{N}\sum_i  \sigma_i
% \end{equation}
% Then, we can write:
% \begin{equation}
%     H(\underline{\sigma}) = \frac{1}{2}N- \frac{1}{2}N m(\underline{\sigma})^2 - NB m(\underline{\sigma})
% \end{equation}




\subsection{Mean-field approximation of the Ising model in higher dimensions}
While a closed-form solution for the Ising model in two dimensions exists, it does not for $d\geq 3$ so we will study its mean-field approximation. This method is actually very general and it can be applied to a more general Ising model, and then be reconduced to the original one. The hamiltonian we focus on is:
\begin{equation}
    H(\underline{\sigma}) = -\sum_{\langle i,j\rangle}J_{i,j}\sigma_i\sigma_j - \sum_i B_i\sigma_i
\end{equation}
Which differs from the standard Ising model by having arbitrary $J_{i,j}$ and $B_i$ for every $i,j$. The idea is to approzimate the Boltzmann distribution $P(\underline{\sigma}) = (1/Z)e^{-\beta H(\underline{\sigma})}$ with a probability with independent variables $Q(\underline{\sigma})= \prod_{i=1}^Nq_i(\sigma_i)$. The idea is to find the $q_i$ such that the ``Distance'' between $P$ and $Q$ is minimized. We will use the Kullback-Leibler divergence as notion of distance.
\begin{definition}
    Given $p(x)$ and $q(x)$ probability distributions over the same finite space $\mathcal{X}$, the Kullback–Leibler (KL) divergence between them is:
    $$D(q||p) = \sum_{x\in\mathcal{X}}q(x)\log\frac{q(x)}{p(x)}$$
\end{definition}
\begin{remark}
    \hfill
    \begin{enumerate}
        \item $D(q||p)$ is convex in $q(x)$.
        \item $D(q||p)\geq 0$ with equality $ \iff p(x)=q(x) \;\forall x\in\mathcal{X}$.
        \item In general, the KL divergence is not symmetric.
    \end{enumerate}
    Then, the KL divergence lacks the symmetry property to be properly defined as a distance between probability distributions.
\end{remark}
We will define $Q$ as the most general joint binary probability distribution:
\begin{equation}
    Q(\underline{\sigma}) = \prod_{i=1}^{N}q_i(\sigma_i); \quad
    q_i(\sigma_i)=\frac{1+m_i\sigma_i}{2}
\end{equation}
Where $m_i$ is the mean of each $q_i$, and it is the parameter which we want to find. Then,
\begin{equation}
    \begin{gathered}
        D(Q||P) = \sum_{\underline{\sigma}\in\mathcal{X}}Q(\underline{\sigma})\log\frac{Q(\underline{\sigma})}{P(\underline{\sigma})}\\
        =\sum_{\underline{\sigma}\in\mathcal{X}}Q(\underline{\sigma})\log Q(\underline{\sigma}) - \sum_{\underline{\sigma}\in\mathcal{X}}Q(\underline{\sigma})\log P(\underline{\sigma}) = (A) + (B)\\
    \end{gathered}
\end{equation}
We can split this in the first term, depending only on $Q$, and the second term, depending on $P$ as well. Then:
\begin{equation}
    \begin{gathered}
        (A) = \sum_{\underline{\sigma}\in\mathcal{X}}Q(\underline{\sigma})\log Q(\underline{\sigma}) = \sum_{i=1}^{N}\left(\frac{1+m_i}{2} \log \frac{1+m_i}{2}+\frac{1-m_i}{2} \log \frac{1-m_i}{2}\right)\\
        (B) = \beta \sum_{i<j} J_{i j} m_i m_j + \beta \sum_i B_i m_i-\log Z
    \end{gathered}
\end{equation}
In the second term, we see that the $\log Z$ term is independent of $m_i$, so we can ignore it. Then, we are interested in finding the values of $m_i$ that solve:
\begin{equation}
    \frac{\partial D(Q||P)}{\partial m_i} = 0 \iff \frac{1}{2} \log \frac{1+m_i}{1-m_i} - \beta \sum_{j\in \partial_i} J_{i j} m_j - \beta B_i = 0
\end{equation}
Then, we find the mean field equation:
\begin{equation}
    m_i = \tanh\left(\beta\sum_{j\in \partial_i} J_{i j} m_j + \beta B_i\right)
\end{equation}
Now, going back to the original Ising model, we can set $J_{i,j} = J$ and $B_i=B \implies m_i = m$. Then, we have the mean field equation for the Ising model in $d$ dimensions:
\begin{equation}
    m = \tanh(\beta (B + 2dJm))
\end{equation}
Let us consider the case $B=0$. Then, depending on the value of $\beta$, we can have one or three solutions to the mean-field equation, depending on the slope of the function $f(m) = \tanh(\beta 2dJm)$. The critical value of $\beta$ is then:
\begin{equation}
    \beta_c = \frac{1}{2dJ}
\end{equation}

\section{The Bornholdt model}
\subsection{Financial motivation}
\textcolor{red}{To be added!}\\
\subsection{Model definition}
\textcolor{red}{Improve}\\
The Bornholdt model is a simple model of a stock market, where agents can buy or sell a stock. The idea is to have a simple model that can capture the strategic interaction between agents in the market. Consider a model with N spins with orientations $\sigma_i\in\{-1,+1\}$, representing the decision of agent $i$ to buy or sell a stock. The local field of the model is updated with heat-bath dynamics according to:
\begin{equation}
    \begin{aligned}
        \sigma_i(t+1) &= +1 \;\text{with }\; p = \frac{1}{1+e^{-2\beta h_i(t)}}\\
        \sigma_i(t+1) &= -1 \;\text{with }\; 1-p
    \end{aligned}
\end{equation}
Where $h_i(t)$ is the local field of agent $i$ at time $t$:
\begin{equation}
    h_i(t)=\sum_{j=1}^N J_{i j} \sigma_j-\alpha C_i(t) \frac{1}{N} \sum_{j=1}^N \sigma_j(t)
\end{equation}
Where $J_{i j}$ is the coupling between agents $i$ and $j$, $\sigma_j$ is the agent's action at $t$, $C_i(t)$ is the strategy of $i$ at time $t$, and $\alpha$ is a parameter. The strategy is updated according to:
\begin{equation}
    C_i(t+1) = -C_i(t) \quad\text{if} \; \alpha\sigma_i(t)C_i(t) \sum_{j=1}^N \sigma_j(t)< 0
\end{equation}