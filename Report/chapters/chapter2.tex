\chapter{Bornholdt model}\label{ch:chapter2}
We now present the model proposed by Bornholdt in \cite{bornholdt}. The idea is to formulate a model with maximum simplicity, which includes the possibility of strategic interaction in the market. The model is based on the Ising model, which is a model of ferromagnetism in statistical mechanics.

\section{Theoretical background: the Ising model}
The Ising model is a simple mathematical model of ferromagnetic materials. In its description, we will mostly follow the notation and tools presented in \cite{mezard_book}. It consists of Ising spins (that is, spins which can take binary values) on a d-dimensional cubic lattice. Mathematically, given a cubic lattice $\mathbb{L}=\{1,\dots,L\}^n$, we define an an Ising spin $\sigma_i\in\{-1,1\}$ for each site $i\in\mathbb{L}$. Then, we can have any configuration $\underline{\sigma} = (\sigma_1,\dots,\sigma_n) \in \mathcal{X}_N=\{+1,-1\}^{\mathbb{L}}$. The energy of a configuration $\underline{\sigma}$ is given by:
\begin{equation}
    H(\underline{\sigma}) = -\sum_{\langle i,j\rangle}\sigma_i\sigma_j - B\sum_i \sigma_i
\end{equation}
Where the sum over $\langle i,j\rangle$ is a sum over all nearest neighbors, and $B$ is an external magnetic field. At equilibrium, the probability of a configuration $\underline{\sigma}$ is given by the Boltzmann distribution:
\begin{equation}
    P(\underline{\sigma}) = \frac{e^{-\beta H(\underline{\sigma})}}{Z}
\end{equation}
Where $\beta$ is the inverse temperature, and $Z$ is the partition function:
\begin{equation}
    Z = \sum_{\underline{\sigma}\in\mathcal{X}_N}e^{-\beta H(\underline{\sigma})}
\end{equation}
Interestingly, despite its simplicity, an analytical solution has been found only in the $d=1$ and $d=2$ cases. Higher dimensions remain unsolved, but numerical methods can be used to study the model in these cases.

One important quantity in the Ising model is the magnetization, which is defined as:
\begin{equation}
    m = \frac{1}{N}\sum_i \langle \sigma_i \rangle
\end{equation}
where $\langle \cdot \rangle$ denotes the average.

\subsection{Solution of the Ising model in the one-dimensional case}
For simplicity, assume $B=0$. In the one-dimensional case, the Ising model can be solved exactly. Recall that:
\begin{equation}
    H(\underline{\sigma}) = -\sum_{\langle i,j\rangle}\sigma_i\sigma_j
\end{equation}
Then, the partition function is given by:
\begin{equation}
    Z = \sum_{\underline{\sigma}\in\mathcal{X}_N}e^{-\beta H(\underline{\sigma})} =
    \sum_{\underline{\sigma}\in\mathcal{X}_N}e^{\beta\sum_{\langle i,j\rangle}\sigma_i\sigma_j}
\end{equation}
Since each spin is connected to its nearest neighbors, we can write:
\begin{equation}
    Z =  \sum_{\underline{\sigma}\in\mathcal{X}_N}e^{\beta\sum_n\sigma_n\sigma_{n+1}}
\end{equation}
Let us define $\tau_n = \sigma_{n-1}\sigma_{n} \implies \sigma_n = \tau_n\tau_{n-1}\dots\tau_2\sigma_1$. Then, we can write:
\begin{equation}
    \begin{gathered}
    Z =  \sum_{\sigma_1\in\{-1,1\}}\sum_{\tau_2,\dots\tau_n}e^{\beta\sum_n\tau_n}
    = 2\sum_{\tau_2,\dots,\tau_N}e^{\beta\sum_n\tau_n}\\
    = 2\sum_{\tau_2,\dots,\tau_N}\prod_n e^{\beta\tau_n}
    = 2(\sum_{\tau_2}e^{\beta\tau_2})\dots(\sum_{\tau_N}e^{\beta\tau_N})\\
    = 2(2\cosh(\beta))^{N}
    \end{gathered}
\end{equation}
Thus, we have found an analytical expression for the partition function in the one-dimensional case. The magnetization can be computed as:
\begin{equation}
    m = \frac{1}{N}\sum_i \langle \sigma_i \rangle = \frac{1}{N}\frac{\partial}{\partial \beta}\log Z = \tanh(\beta)
\end{equation}
%check magnetization formula

\subsection{The Curie-Weiss model}
The Curie-Weiss model is remarkably similar to the Ising model, but with all the spins interacting with each other, then, the model represents a fully connected graph of spins rather than a lattice. The Hamiltonian is given by:
\begin{equation}
    H(\underline{\sigma}) = -\frac{1}{N}\sum_{i,j}\sigma_i\sigma_j - B\sum_i \sigma_i
\end{equation}
The scaling factor $1/N$ is introduced to have a non-trivial free energy. This model is interesting because it introduced the concept of mean-field approximations. To compute the partition function, we first notice that the empirical magnetization is:
\begin{equation}
    m(\underline{\sigma}) = \frac{1}{N}\sum_i  \sigma_i
\end{equation}
Then, we can write:
\begin{equation}
    H(\underline{\sigma}) = \frac{1}{2}N- \frac{1}{2}N m(\underline{\sigma})^2 - NB m(\underline{\sigma})
\end{equation}



\subsection{Mean-field approximation of the Ising model in the two-dimensional case}