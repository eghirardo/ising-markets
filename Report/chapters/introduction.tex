\chapter{Introduction}\label{ch:introduction}
The relationship between physics and finance is a long-standing one. \cite{bachelier} is perhaps the first example of a model developed by physicists (Brownian motion) being applied to financial markets, specifically to the pricing of derivative products. This work would remain largely unnoticed until \cite{black_scholes} was published, now considered the foundation of modern quantitative finance.

Beyond derivative pricing, econophysics has been active since the 1990s, applying methods from statistical physics to a wide range of problems in economics and finance. For example, \cite{bouchaud_mezard_2000} examines the distribution of wealth in a simplified economy, mapping it to the random `directed polymer' problem.

Econophysics emerged as a distinct interdisciplinary field when physicists began to systematically apply concepts from statistical mechanics, complex systems, and nonlinear dynamics to economic and financial systems. The motivation comes from the observation that financial markets, like physical systems, are composed of many interacting agents whose collective behavior gives rise to emergent phenomena such as bubbles, crashes, and fat-tailed return distributions.

For an introduction to econophysics, see \cite{Mantegna_Stanley_1999} and \cite{econophysics_2011_review}.

The scope of this thesis is to explore how concepts and models from statistical physics, particularly spin models, can be used to gain insight into the collective behavior observed in financial markets. After introducing the necessary background in financial mathematics and statistical mechanics, we focus on the Bornholdt model, a spin-based agent model that captures both trend-following and contrarian strategies. Through analytical discussion and numerical simulations, we investigate how this model reproduces key stylized facts of financial time series, such as volatility clustering and fat-tailed return distributions, highlighting the strengths and limitations of physics-inspired approaches to financial market dynamics.